\chapter{Pendahuluan}

Bab Pendahuluan secara umum yang dijadikan landasan kerja dan arah kerja penulis tugas akhir, berfungsi mengantar pembaca untuk membaca laporan tugas akhir secara keseluruhan.

\section{Latar Belakang}
\label{sec:latarbelakang}

Dalam dekade terakhir, skala \textit{dataset} pelatihan untuk pembelajaran mendalam telah mengalami pertumbuhan signifikan. Dimulai dari ImageNet-1K yang terdiri atas 1,28 juta gambar JPEG ($\approx$150 GigaByte (GB)) \parencite{AcceleratingDL}, hingga ke \textit{dataset} video besar seperti YouTube-8M milik Google yang mencapai 1,53 TeraByte (TB). Pertumbuhan ini membawa tantangan baru dalam manajemen dan distribusi data, khususnya dalam sistem pelatihan terdistribusi.

Untuk menangani tantangan tersebut, banyak sistem pelatihan skala besar menggunakan \textit{Network File System} (NFS) sebagai lapisan penyimpanan bersama, karena kemudahan dalam manajemen data, pengaturan versi, serta akses oleh banyak simpul komputasi secara simultan. Namun, NFS pada dasarnya dirancang untuk beban kerja akses berkas konvensional, bukan untuk pola akses \textit{high-concurrency}, \textit{bursty}, dan tidak berurutan yang umum dalam beban kerja pembelajaran mendalam \parencite{360survey}.

Studi oleh Pumma dkk.(2019) menunjukkan bahwa operasi \textit{Input/Output} (I/O) dapat mendominasi waktu pelatihan, dengan hingga 90\% dari total waktu \textit{epoch} dihabiskan hanya untuk memuat data. Meskipun kerangka kerja pembelajaran mendalam modern mendukung pemuatan data secara paralel, efisiensinya sering terhambat oleh keterbatasan \textit{file system} yang mendasarinya -- khususnya pada mekanisme \textit{caching} dan \textit{prefetching} tradisional yang tidak cocok dengna pola pengaksesan pembelajaran mendalam yang acak \parencite{BundleShuffle}.

Kondisi ini menyoroti kebutuhan mendesak akan pengembangan sistem \textit{cache} maupun \textit{prefetcher} yang lebih adaptif dan selaras dengan karakteristik beban kerja pembelajaran mendalam. Namun, pengembangan tersebut saat ini terkendala oleh ketiadaan \textit{dataset trace} I/O yang merekam pola akses data riil pada skenario pelatihan pembelajaran mendalam. Tanpa data tersebut, sulit untuk merancang, menguji, dan membandingkan efektivitas berbagai strategi \textit{caching} dan \textit{prefetching}. Oleh karena itu, tugas akhir ini berfokus pada pengembangan \textit{dataset trace} I/O pada NFS, yang dikumpulkan dari skenario pelatihan pembelajaran mendalam menggunakan berbagai strategi \textit{shuffling}. \textit{Dataset} ini harapannya dapat menjadi fondasi bagi penelitian dan pengembangan \textit{cache} maupun \textit{prefetcher} yang lebih cerdas dan efisien di masa depan.

\section{Rumusan Masalah}

Rumusan masalah yang hendak dijawab pada tugas akhir ini adalah:

\begin{enumerate}
    \item Bagaimana format dan struktur \textit{dataset trace} yang dikembangkan?
    \item Bagaimana cara merekam aktivitas I/O di tingkat NFS?
    \item Apa karakteristik pola akses I/O pada beban kerja pelatihan mendalam dengan berbagai strategi \textit{shuffling}?
\end{enumerate}

\section{Tujuan}

Tujuan dari tugas akhir ini adalah:

\begin{enumerate}
    \item Mengembangkan \textit{dataset trace} I/O NFS yang merekam pola akses data pada skenario pelatihan pembelajaran mendalam
    \item Merancang format dan struktur \textit{dataset trace} I/O yang informatif, terstandarisasi, dan dapat digunakan oleh peneliti untuk mengevaluasi dan mengembangan strategi \textit{caching} dan \textit{prefetching}
    \item Mengembangkan dan mengimplementasikan metode \textit{tracing} aktivitas I/O di tingkat NFS
    \item Menganalisis dan membandingkan karakteristik pola akses I/O yang dihasilkan dari pelatihan pembelajaran mendalam dengan menerapkan berbagai strategi \textit{shuffling}, yaitu \textit{global shuffling}, \textit{buffered shuffling}, dan \textit {bundle shuffling}
\end{enumerate}

\section{Batasan Masalah}

Batasan masalah pada tugas akhir ini adalah:

\begin{enumerate}
    \item Fokus pada pelatihan pembelajaran mendalam menggunakan \textit{dataset} gambar, khususnya pada \textit{dataset} FireRisk
    \item Pengumpulan \textit{dataset trace} I/O hanyya dilakukan pada proses pelatihan satu klien saja, tidak pada skenario pelatihan terdistribusi
    \item Tidak mencakup pengembangan atau evaluasi strategi \textit{caching} dan \textit{prefetching} yang lebih kompleks, fokus pada pengumpulan data dan analisis pola akses I/O
\end{enumerate}


\section{Metodologi}

Berikut merupakan tahapan yang dilaksanakan dalam tugas akhir ini:

\begin{enumerate}
    \item Menyiapkan lingkungan pengembangan yang diperlukan, termasuk sistem NFS dan perangkat keras yang akan digunakan untuk pelatihan pembelajaran mendalam
    \item Menyiapkan \textit{dataset} yang akan digunakan, yaitu \textit{dataset} FireRisk dalam format folder PNG
    \item Mengimplementasikan sistem \textit{tracing} I/O di tingkat NFS untuk merekam aktivitas akses berkas selama pelatihan pembelajaran mendalam
    \item Mengimplementasikan beban kerja pembelajaran mendalam dengan strategi \textit{shuffling} yang berbeda, yaitu \textit{global shuffling}, \textit{buffered shuffling}, dan \textit{bundle shuffling}, pada proses pelatihan pembelajaran mendalam
    \item Menjalankan pelatihan pembelajaran mendalam dengan berbagai strategi \textit{shuffling} dan mengumpulkan \textit{dataset trace} I/O yang dihasilkan
    \item Menganalisis \textit{dataset trace} I/O yang dikumpulkan untuk mengidentifikasi pola akses data, karakteristik beban kerja, dan perbandingan antara strategi \textit{shuffling} yang diterapkan
\end{enumerate}

\section{Sistematika Pembahasan}
% TODO: tambahkan sistematika pembahasan
Subbab ini berisi penjelasan ringkas isi per bab. Penjelasan ditulis dalam satu paragraf per bab buku.
